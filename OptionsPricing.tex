\documentclass[12pt]{article}
\usepackage{lingmacros}
\usepackage{tree-dvips}
\begin{document}

\title{Options Pricing Research }
\author{Soham Deshpande}
\maketitle
\clearpage
\tableofcontents
\clearpage

\section{Options}
Options are a versatile financial product that are based on the value of the 
underlying security. These contracts offers the buyer an opportunity to buy or
sell, but unlike a future, the contact holder is not required to buy or sell the 
commodity.Through paying a premium, a buyer can gain the rights granted by the
contract.
\\
There are two types of options: call and put options. A call options is the
right to buy
or take a long position in a given asset. A put options is the right to sell or
take a short position in a given asset. The asset to be bought or solder under
the terms of the options is the underlying asset. The price at which the
underlying will be delivered is called the Strike Price. The date after which the
option may no longer be exercised is the expiration date.
\\
The contact specifications contain the following: Underlying asset, expriation
date, exercise price and type. 
 
\section{Binomial Pricing Model}
The binomial asset pricing model is a powerful tool that can be used to
understand arbitrage pricing theory. This first section will explore the
simplest binomial model before generalising to a more realistic, complex
multiperiod binomial model. 
\\
For the general one-period model we can call the beginning of the period time
zero($T_{0}$) and the end of the period time one($T_1$). At time zero, we have a
stock whose price per share can be written as $S_0$, a positive quantity known
at $T_0$. At $T_1$, the price per share of this stock will be one of two
positive values, which we can denote as $S_1(H)$ and $S_1(T)$, the S and H
standing for heads and tails respectively. We can assume that the probability of
a head is $p$, therefore making the probability of a tail $q = 1-p$, both being
positive values. 
\\
The outcome of the coin toss and hence the value which the stock price will take
at time one is known at $T_1$ but not at $T_0$. Any quantity not known at $T_0$
will be known as random henceforth as it depends on the random experiment of
tossing a coin. \\
Two new positive values can be introduced: \\
$u = \frac{S_1(H)}{S_0)},  d = \frac{S_1(T)}{S_0}$.
\\ It is assumed that $d<u$ where u can be referred to as the up factor and d as
the down factor.
\\
An interest rate $r$ can be introdcued. One dollar invested in the market at
$T_0$ will yield 1+r dollars at $T_1$. 
\\
An essential feature of an efficient market is that if a trading strategy can
turn nothing into something, then it must also run the risk of loss. Otherwise
there would be an arbitrage. More specifically, an arbitrage can be defined as
as a trading strategy that begins with no money, has zero probability of losing
money, and has a positive probability of making money. 
\\
In the one-period binomial model, to rule out arbitrage we must assume
$0<d<1+r<u$.
\\
Let us now consider a European call option which relies on the fact that its
owner has the right but not the obligation to buy one share of the stock at 
$T_1$for the strike price K. We shall assume here that $S_1(T)<K<S_1(H)$. If we
get a tail then the option expires worthless. If the outcome is a head then the
option can be exercised and yields a profit of $S_1(H) - K$. This can be
summarised by saying that the option at $T_1$ is worth $(S_1 - K)^+$
\end{document}
